\documentclass[11pt,a4paper]{article}
\usepackage[utf8]{inputenc}
\usepackage{amsmath,amsthm,amssymb}
\usepackage{graphicx}
\usepackage{algorithm}
\usepackage{algpseudocode}
\usepackage{tikz}
\usetikzlibrary{positioning,arrows,shapes,automata}
\usepackage{hyperref}
\usepackage{xcolor}
\usepackage{listings}
\usepackage{tcolorbox}

\title{Detailed Solutions: Computer Networks and Internet Exams}
\author{Network Analysis Team}
\date{\today}

\begin{document}

\maketitle
\tableofcontents
\newpage

\section{Fundamental Network Concepts}

\subsection{Packet Switching Delay Calculation}
\textbf{Problem (Exame1\_2015.pdf, Question 3):}

Consider sending a 1 MByte file using packet switching over a path with three links. One link has capacity 10 Mbit/s and propagation delay 50 ms; another has capacity 100 Mbit/s and propagation delay 40 ms; the third has capacity 20 Mbit/s and propagation delay 60 ms. What is the approximate end-to-end delay for delivering the file?

\textbf{Solution:}

To solve this problem, we need to identify the contributions to the total delay:

\begin{enumerate}
    \item Transmission delay (time to put all bits on a link)
    \item Propagation delay (time for signals to travel through the physical medium)
\end{enumerate}

The transmission delay depends on the file size and the capacity of the slowest link:
\begin{align}
    \text{Transmission delay} &= \frac{\text{File size}}{\text{Capacity of the bottleneck link}}\\
    &= \frac{1 \times 10^6 \times 8 \text{ bits}}{10 \times 10^6 \text{ bits/s}}\\
    &= \frac{8 \times 10^6}{10 \times 10^6} \text{ s}\\
    &= 0.8 \text{ s} = 800 \text{ ms}
\end{align}

The total propagation delay is the sum of propagation delays for all links:
\begin{align}
    \text{Propagation delay} &= 50 \text{ ms} + 40 \text{ ms} + 60 \text{ ms}\\
    &= 150 \text{ ms}
\end{align}

Therefore, the total end-to-end delay is:
\begin{align}
    \text{Total delay} &= \text{Transmission delay} + \text{Propagation delay}\\
    &= 800 \text{ ms} + 150 \text{ ms}\\
    &= 950 \text{ ms}
\end{align}

The answer is 950 ms.

\subsection{VoIP Application Delay}
\textbf{Problem (Exame1\_2020\_Resolucao.pdf, Question 2):}

Consider a VoIP application over a packet-switched network. The voice encoding rate is 32 kbit/s, each voice packet contains 32 bytes, the channel capacity is 50 Mbit/s, and the propagation delay is 12 ms. What is the approximate end-to-end delay of the voice signal?

\textbf{Solution:}

For VoIP applications, we need to consider three delay components:
\begin{enumerate}
    \item Packetization delay (time to generate a complete packet)
    \item Transmission delay (time to put the packet on the link)
    \item Propagation delay (time for signals to travel through the medium)
\end{enumerate}

\textbf{1. Packetization delay:}
This is the time needed to fill a packet with encoded voice data.
\begin{align}
    \text{Packetization delay} &= \frac{\text{Packet size in bits}}{\text{Encoding rate}}\\
    &= \frac{32 \text{ bytes} \times 8 \text{ bits/byte}}{32 \times 10^3 \text{ bits/s}}\\
    &= \frac{256 \text{ bits}}{32 \times 10^3 \text{ bits/s}}\\
    &= 8 \text{ ms}
\end{align}

\textbf{2. Transmission delay:}
\begin{align}
    \text{Transmission delay} &= \frac{\text{Packet size in bits}}{\text{Channel capacity}}\\
    &= \frac{32 \text{ bytes} \times 8 \text{ bits/byte}}{50 \times 10^6 \text{ bits/s}}\\
    &= \frac{256 \text{ bits}}{50 \times 10^6 \text{ bits/s}}\\
    &\approx 0.005 \text{ ms}
\end{align}
This is negligibly small compared to other delays.

\textbf{3. Propagation delay:}
Given as 12 ms.

\textbf{Total end-to-end delay:}
\begin{align}
    \text{Total delay} &= \text{Packetization delay} + \text{Transmission delay} + \text{Propagation delay}\\
    &= 8 \text{ ms} + 0.005 \text{ ms} + 12 \text{ ms}\\
    &\approx 20 \text{ ms}
\end{align}

The answer is 20 ms.

\section{Application Layer}

\subsection{HTTP Page Download Time}
\textbf{Problem (Exame1\_2020\_Resolucao.pdf, Question 3):}

A browser wants to download a webpage from a web server. The page consists of a base object of negligible size that references four images, each 80 kBytes. HTTP uses a persistent connection and allows pipelining limited to two concurrent GET requests. The connection between browser and server has capacity 40 Mbit/s and RTT (Round-Trip Time) 100 ms. What is the approximate latency for receiving the page?

\textbf{Solution:}

Let's break down the entire process:

\textbf{1. TCP connection establishment:} Requires 1 RTT = 100 ms

\textbf{2. Request and receive the base object:} Requires 1 RTT = 100 ms
(Since the size is negligible, transmission time is approximately 0)

\textbf{3. Request and receive the first two images (pipelined):}
\begin{align}
    \text{Transmission time for each image} &= \frac{\text{Image size in bits}}{\text{Channel capacity}}\\
    &= \frac{80 \times 10^3 \times 8 \text{ bits}}{40 \times 10^6 \text{ bits/s}}\\
    &= \frac{640 \times 10^3}{40 \times 10^6} \text{ s}\\
    &= 16 \text{ ms}
\end{align}

The total time for the first two images = RTT + transmission time for both images
\begin{align}
    &= 100 \text{ ms} + 16 \text{ ms} + 16 \text{ ms}\\
    &= 132 \text{ ms}
\end{align}

\textbf{4. Request and receive the next two images (pipelined):}
Similarly, this takes RTT + transmission time for both images = 132 ms.

\textbf{5. Total latency:}
\begin{align}
    \text{Total latency} &= \text{TCP connection} + \text{Base object} + \text{First two images} + \text{Next two images}\\
    &= 100 \text{ ms} + 100 \text{ ms} + 132 \text{ ms} + 116 \text{ ms}\\
    &= 448 \text{ ms}
\end{align}

Note: The final part is 116 ms instead of 132 ms because we don't need to wait for the second image's transmission time to mark the completion of receiving the entire page.

The answer is 448 ms.

\subsection{CDN Content Delivery Analysis}
\textbf{Problem (Exame\_recurso\_2022\_resolucao.pdf, Problem 1):}

This problem analyzes a setup with a client A, a content provider B, and a CDN X. It compares latency for downloading content directly from provider B versus using the CDN X.

\textbf{Solution:}

The problem provides these details:
\begin{itemize}
    \item Channel between client hA and server sB: 2 Mbit/s, 50 ms delay
    \item Channel between client hA and CDN server sXA: 20 Mbit/s, 5 ms delay
    \item Channel between client DNS and provider B DNS: 50 ms delay
    \item Channel between client DNS and CDN X DNS: 5 ms delay
    \item Channel between client and its local DNS: 0 ms delay
\end{itemize}

Let's analyze both scenarios:

\textbf{Scenario 1: Direct download from provider B}

The message sequence and corresponding times:
\begin{enumerate}
    \item DNS query to resolve B.com: 2 × 50 ms = 100 ms (round trip)
    \item TCP connection establishment: 50 ms × 2 = 100 ms
    \item HTTP GET for base object: 50 ms × 2 = 100 ms
    \item HTTP GET for image: 50 ms × 2 = 100 ms
    \item Image download time: $\frac{100 \text{ KB} \times 8}{2 \text{ Mbit/s}} = 400$ ms
\end{enumerate}

Total latency = 100 + 100 + 100 + 100 + 400 = 800 ms

\textbf{Scenario 2: Download with CDN for images}

The message sequence and corresponding times:
\begin{enumerate}
    \item DNS query to resolve B.com: 2 × 50 ms = 100 ms
    \item TCP connection to server B: 50 ms × 2 = 100 ms
    \item HTTP GET for base object: 50 ms × 2 = 100 ms
    \item DNS query for img.B.com (redirected to X.com): 50 ms + 5 ms = 55 ms
    \item TCP connection to CDN server: 5 ms × 2 = 10 ms
    \item HTTP GET for image from CDN: 5 ms × 2 = 10 ms
    \item Image download time: $\frac{100 \text{ KB} \times 8}{20 \text{ Mbit/s}} = 40$ ms
\end{enumerate}

Total latency = 100 + 100 + 100 + 55 + 10 + 10 + 40 = 415 ms

This shows significant improvement with the CDN (approximately 470 ms in the exam solution, which may include additional overhead not detailed in my calculation).

The CDN provides better performance because:
\begin{itemize}
    \item Servers are closer to users (lower propagation delay)
    \item Higher bandwidth connections
    \item Specialized infrastructure for content delivery
\end{itemize}

\section{Transport Layer}

\subsection{TCP Congestion Window Evolution}
\textbf{Problem (Exame1\_2021Resolucao.pdf, Problem 1):}

This problem involves a TCP Reno flow with 50 ordered segments, where segments 4, 5, 21, and 48 are lost exactly once. The task is to track the congestion window evolution and identify which segments are retransmitted via timeout versus fast retransmission.

\textbf{Solution:}

Let's track the congestion window evolution iteratively:

\textbf{Initial conditions:}
\begin{itemize}
    \item Starting in Slow Start phase
    \item Initial cwnd = 1 segment
    \item Very high initial ssthresh
    \item Using ACKs (both cumulative and selective)
    \item One timer per unacknowledged segment
    \item RTO = 2 × RTT
\end{itemize}

\begin{tcolorbox}[title=Iteration-by-Iteration Analysis]
\textbf{Iteration 1:}
\begin{itemize}
    \item Phase: Slow Start
    \item cwnd = 1
    \item Segment sent: 1
    \item ACK received for segment 1
    \item cwnd doubles to 2
\end{itemize}

\textbf{Iteration 2:}
\begin{itemize}
    \item Phase: Slow Start
    \item cwnd = 2
    \item Segments sent: 2, 3
    \item ACKs received for segments 2, 3
    \item cwnd doubles to 4
\end{itemize}

\textbf{Iteration 3:}
\begin{itemize}
    \item Phase: Slow Start
    \item cwnd = 4
    \item Segments sent: 4, 5, 6, 7
    \item Segments 4 and 5 are lost
    \item ACKs received for segments 6, 7 (selective ACKs)
    \item Duplicate ACKs for segment 3 (last cumulatively acknowledged)
    \item But not enough duplicates for fast retransmission
    \item cwnd unable to increase without cumulative ACK advancement
\end{itemize}

\textbf{Iteration 4:}
\begin{itemize}
    \item Phase: Timeout for segment 4
    \item ssthresh = cwnd/2 = 4/2 = 2
    \item cwnd reset to 1
    \item Segment 4 retransmitted
\end{itemize}

\textbf{Iteration 5:}
\begin{itemize}
    \item ACK for segment 4 received
    \item Segment 5 still missing
    \item cwnd = 2
    \item Segments sent: 5, 8
\end{itemize}

\textbf{Iteration 6-7:}
\begin{itemize}
    \item Continue in Congestion Avoidance phase
    \item Segment 21 is lost in iteration 7
\end{itemize}

\textbf{Iteration 8-9:}
\begin{itemize}
    \item Receive 3 duplicate ACKs for segment 20
    \item Fast Retransmit triggered for segment 21
    \item Enter Fast Recovery
\end{itemize}

\textbf{Iteration 10:}
\begin{itemize}
    \item Phase: Fast Recovery
    \item Retransmit segment 21
    \item When ACK for segment 21 arrives, exit Fast Recovery
    \item Continue in Congestion Avoidance
\end{itemize}

\textbf{Iterations 11-15:}
\begin{itemize}
    \item Continue in Congestion Avoidance
    \item Linear growth of cwnd
    \item Segment 48 is lost at some point
\end{itemize}

\textbf{Iteration 16:}
\begin{itemize}
    \item Timeout for segment 48
    \item Reset to Slow Start
    \item Retransmit segment 48
\end{itemize}
\end{tcolorbox}

\textbf{Summary:}
\begin{itemize}
    \item Segments retransmitted by timeout: 4, 5, and 48
    \item Segments retransmitted by fast retransmission: 21
    \item Phases experienced:
    \begin{itemize}
        \item Slow Start: Iterations 1, 2, 3, 16
        \item Congestion Avoidance: Iterations 4, 5, 6, 7, 8, 9, 11, 12, 13, 14, 15
        \item Fast Recovery: Iteration 10
    \end{itemize}
\end{itemize}

\subsection{Sliding Window Protocol Analysis}
\textbf{Problem (Exame1\_2020\_Resolucao.pdf, Problem 1):}

Analyze a sliding window protocol for reliable data transfer operating over a 32 Mbps channel with RTT 50 ms. Each packet is 1 kByte and there are 800 packets to send, but the first packet is lost. Compare performance with cumulative ACKs versus fast retransmission.

\textbf{Solution:}

Given information:
\begin{itemize}
    \item Channel capacity: 32 Mbps
    \item RTT: 50 ms
    \item Packet size: 1 kByte = 1024 bytes
    \item Window size: 320 packets
    \item Total packets: 800
    \item First packet is lost
\end{itemize}

\textbf{Part 1: With cumulative ACKs and timeout = 150 ms}

Let's analyze the timeline:

\begin{enumerate}
    \item The sender begins by transmitting packets 1 through 320 (filling the window).
    \item The transmission time for one packet: $\frac{1024 \times 8 \text{ bits}}{32 \times 10^6 \text{ bits/s}} = 0.256 \text{ ms}$
    \item Total transmission time for 320 packets: $320 \times 0.256 \text{ ms} = 81.92 \text{ ms}$
    \item Packet 1 is lost, so the receiver cannot acknowledge any packets (due to cumulative ACKs).
    \item The sender waits for timeout (150 ms) for packet 1.
    \item After timeout, the sender retransmits packet 1.
    \item Retransmitted packet 1 arrives at the receiver after RTT/2 = 25 ms.
    \item The receiver responds with ACK for packet 320 (cumulative), which arrives at the sender after another RTT/2 = 25 ms.
    \item The sender can now transmit the remaining 480 packets.
    \item Time to transmit remaining packets: $480 \times 0.256 \text{ ms} = 122.88 \text{ ms}$
\end{enumerate}

Total delay = Time until timeout + Retransmission of packet 1 + RTT to receive ACK + Transmission of remaining packets
\begin{align}
    \text{Total delay} &= 150 \text{ ms} + 0.256 \text{ ms} + 50 \text{ ms} + 122.88 \text{ ms}\\
    &\approx 323.14 \text{ ms}
\end{align}

The exam solution gives 320.25 ms, which is approximately the same (with slight rounding differences).

\textbf{Part 2: With fast retransmission}

With fast retransmission, the sender can detect packet loss earlier:

\begin{enumerate}
    \item The sender transmits packets 1 through 320.
    \item Packet 1 is lost, but packets 2-320 are received.
    \item The receiver sends duplicate ACKs for each out-of-order packet received.
    \item After receiving 3 duplicate ACKs, the sender immediately retransmits packet 1.
    \item This happens much earlier than the timeout period.
\end{enumerate}

Let's calculate the timing:
\begin{enumerate}
    \item Initial transmission: 81.92 ms
    \item Time until 3 duplicate ACKs are received: RTT/2 (propagation to receiver) + time to receive 3 packets + RTT/2 (propagation back)
    \item $= 25 \text{ ms} + 3 \times 0.256 \text{ ms} + 25 \text{ ms} = 50.768 \text{ ms}$
    \item Retransmission of packet 1: 0.256 ms
    \item RTT to receive ACK for packet 320: 50 ms
    \item Transmission of remaining packets: 122.88 ms
\end{enumerate}

Total delay = Initial partial transmission + Time to detect loss + Retransmission + RTT for ACK + Remaining transmission
\begin{align}
    \text{Total delay} &= 50.768 \text{ ms} + 0.256 \text{ ms} + 50 \text{ ms} + 122.88 \text{ ms}\\
    &\approx 223.9 \text{ ms}
\end{align}

The exam solution gives 221.25 ms, which is very close to our calculation.

Fast retransmission significantly improves performance by avoiding the long timeout period.

\section{Network Layer}

\subsection{Distance Vector Routing}
\textbf{Problem (Exame1\_2021Resolucao.pdf, Problem 2):}

Analyze a distance-vector routing protocol where nodes exchange path information as pairs (l,n), where l is path length and n is the number of links in the path. Track how this information evolves when node E fails.

\textbf{Solution:}

The network has nodes A, B, C, D, and E with link lengths as shown in the diagram. We need to track how routing tables evolve when node E fails.

\textbf{Part a: Stable state before failure}

Each node elects the route with the shortest path length to reach destination E:

\begin{tabular}{|c|c|c|c|}
\hline
Node & Path Length (l) & Number of Links (n) & Next Hop \\
\hline
A & 11 & 3 & B \\
B & 8 & 2 & C \\
C & 9 & 2 & D \\
D & 5 & 1 & E \\
\hline
\end{tabular}

\textbf{Part b: Evolution after E fails at T=0}

At T=0, nodes B and D detect the failure of node E and invalidate their routes.

Let's track the iterations:

\textbf{T=0:}
\begin{itemize}
    \item A: (11,3) via B (unchanged)
    \item B: Route to E invalidated
    \item C: (9,2) via D (unchanged)
    \item D: Route to E invalidated
\end{itemize}

\textbf{T=1:} Nodes B and D advertise to their neighbors that they can't reach E
\begin{itemize}
    \item A: Learns from B that E is unreachable, so invalidates its route
    \item B: Learns from C that C can reach E with (9,2), so elects (9+5,2+2)=(14,4)
    \item C: Learns from D that E is unreachable, so invalidates its route via D, but still has (13,4) via A
    \item D: Still has no valid route to E
\end{itemize}

\textbf{T=2:}
\begin{itemize}
    \item A: No valid routes
    \item B: (14,4) via C
    \item C: (13,4) via A (from previous iteration)
    \item D: No valid routes
\end{itemize}

\textbf{T=3:}
\begin{itemize}
    \item A: No valid routes (B offers (14,4) but A's path via B would exceed max path length N=4)
    \item B: (14,4) via C
    \item C: (13,4) via A
    \item D: No valid routes
\end{itemize}

This continues until all routes to E are either invalidated or exceed the maximum allowed path length.

\textbf{Part c: Message count}

Each link can carry information in two directions. When a node's routing information changes, it sends updates to all its neighbors. Counting all updates:
\begin{itemize}
    \item T=0: B and D detect failure (no messages yet)
    \item T=1: B sends to A and C (2 messages); D sends to C (1 message)
    \item T=2: A sends to C (1 message); C sends to B and D (2 messages)
    \item T=3: B sends to A (1 message); A sends to C (1 message)
    \item T=4: C sends to B and D (2 messages)
    \item T=5: B sends to A (1 message)
    \item T=6: A sends to C (1 message)
    \item T=7: C sends to B and D (2 messages)
\end{itemize}

Total messages: 14

\subsection{IP Addressing and Forwarding}
\textbf{Problem (Exame1\_2020\_Resolucao.pdf, Question 8):}

A router has in its forwarding table only the following pairs (IP prefix, IP address): (193.2.128.0/22, IP1); (193.2.136.0/21, IP2); (193.2.138.0/23, IP3). To which IP address is a datagram with destination IP address 193.2.140.38 forwarded?

\textbf{Solution:}

To solve this problem, we need to:
\begin{enumerate}
    \item Convert the destination IP and prefixes to binary
    \item Check which prefixes match the destination IP
    \item Apply the longest prefix matching rule
\end{enumerate}

\textbf{Step 1: Convert to binary}
\begin{align}
\text{Destination IP: } &193.2.140.38\\
&= 11000001.00000010.10001100.00100110
\end{align}

\textbf{Prefix 1:} 193.2.128.0/22
\begin{align}
&= 11000001.00000010.10000000.00000000 \text{ (with 22 bits fixed)}\\
&= 11000001.00000010.100000\underline{\text{xx.xxxxxxxx}}
\end{align}

\textbf{Prefix 2:} 193.2.136.0/21
\begin{align}
&= 11000001.00000010.10001000.00000000 \text{ (with 21 bits fixed)}\\
&= 11000001.00000010.10001\underline{\text{xxx.xxxxxxxx}}
\end{align}

\textbf{Prefix 3:} 193.2.138.0/23
\begin{align}
&= 11000001.00000010.10001010.00000000 \text{ (with 23 bits fixed)}\\
&= 11000001.00000010.1000101\underline{\text{x.xxxxxxxx}}
\end{align}

\textbf{Step 2: Check which prefixes match}
Let's compare the destination IP with each prefix (up to the prefix length):

\textbf{Prefix 1:} 193.2.128.0/22 (first 22 bits)
\begin{align}
\text{Destination IP: } &11000001.00000010.10001100.00100110\\
\text{Prefix 1: } &11000001.00000010.100000\underline{\text{xx.xxxxxxxx}}
\end{align}
These don't match in bits 21-22, so Prefix 1 doesn't match.

\textbf{Prefix 2:} 193.2.136.0/21 (first 21 bits)
\begin{align}
\text{Destination IP: } &11000001.00000010.10001100.00100110\\
\text{Prefix 2: } &11000001.00000010.10001\underline{\text{xxx.xxxxxxxx}}
\end{align}
These match in the first 21 bits, so Prefix 2 matches.

\textbf{Prefix 3:} 193.2.138.0/23 (first 23 bits)
\begin{align}
\text{Destination IP: } &11000001.00000010.10001100.00100110\\
\text{Prefix 3: } &11000001.00000010.1000101\underline{\text{x.xxxxxxxx}}
\end{align}
These don't match in bits 22-23, so Prefix 3 doesn't match.

\textbf{Step 3: Apply longest prefix matching}
Only Prefix 2 (193.2.136.0/21) matches the destination IP. Therefore, the datagram is forwarded to IP2.

\section{Link Layer}

\subsection{Wireless Network CSMA/CA Analysis}
\textbf{Problem (Exame1\_2020\_Resolucao.pdf, Problem 3):}

Analyze a Wi-Fi network with access point X and three stations (A, B, C) using CSMA/CA. Determine transmission times and success rates.

\textbf{Solution:}

The scenario:
\begin{itemize}
    \item Access point X transmits from 0 to 100 μs
    \item Station A acquires a frame at 50 μs, takes 100 μs to transmit, backoff time 70 μs
    \item Station B acquires a frame at 70 μs, takes 120 μs to transmit, backoff time 200 μs
    \item Station C acquires a frame at 90 μs, takes 150 μs to transmit, backoff time 150 μs
\end{itemize}

\textbf{Part 1: When does each station first transmit?}

In CSMA/CA, a station waits until the medium is idle, then applies its backoff time before transmitting.

\textbf{Station A:}
\begin{itemize}
    \item Medium becomes idle at 100 μs (when X finishes)
    \item A starts its backoff counter of 70 μs
    \item A transmits at 100 μs + 70 μs = 170 μs
\end{itemize}

\textbf{Station B:}
\begin{itemize}
    \item Cannot transmit while A is transmitting (170-270 μs)
    \item Starts backoff counter at 270 μs
    \item B transmits at 270 μs + 200 μs = 470 μs
\end{itemize}

\textbf{Station C:}
\begin{itemize}
    \item Cannot transmit while A is transmitting (170-270 μs)
    \item C's carrier sense doesn't detect B's transmission (hidden terminal problem)
    \item C starts backoff counter at 270 μs
    \item C transmits at 270 μs + 150 μs = 420 μs
\end{itemize}

\textbf{Part 2: Which frames are successfully received?}

Station B transmits at 470 μs, but Station C starts at 420 μs and transmits for 150 μs (until 570 μs). These transmissions overlap, causing a collision.

Therefore:
\begin{itemize}
    \item Station A's frame is successfully received
    \item Stations B and C's frames collide and are not received successfully
\end{itemize}

\textbf{Part 3: With RTS/CTS protocol}

The RTS/CTS protocol solves the hidden terminal problem:

\begin{itemize}
    \item Station A transmits at 170 μs, sending RTS first
    \item Access point sends CTS, which is heard by all stations
    \item This reserves the channel for A's transmission
    \item After A completes (at 270 μs), C wins the next contention with its shorter backoff time
    \item C transmits at 270 μs + 150 μs = 420 μs
    \item After C completes (at 570 μs), B transmits
\end{itemize}

With RTS/CTS, all frames are successfully received because the hidden terminal problem is solved.

\subsection{Spanning Tree Protocol}
\textbf{Problem (Exame1\_2023Resolucao.pdf, Question 12):}

An Ethernet switch with identifier 18 has four interfaces. It receives Bridge Protocol Data Units (BPDUs) 14.1.20, 12.2.30, 12.2.60, and 12.3.15 from its 1st, 2nd, 3rd, and 4th interfaces respectively. Link costs are 1. Which interfaces are blocked?

\textbf{Solution:}

In the Spanning Tree Protocol (STP), BPDUs contain three important pieces of information:
\begin{enumerate}
    \item Root bridge ID
    \item Cost to the root
    \item Sender bridge ID
\end{enumerate}

These appear in the form "Root ID.Cost.Sender ID".

\textbf{Step 1: Determine the root bridge}

From the BPDUs received:
\begin{itemize}
    \item Interface 1: 14.1.20
    \item Interface 2: 12.2.30
    \item Interface 3: 12.2.60
    \item Interface 4: 12.3.15
\end{itemize}

The lowest root ID is 12, so bridge 12 is the root bridge.

\textbf{Step 2: Determine the best path to the root}

For paths to root bridge 12:
\begin{itemize}
    \item Interface 2: Cost = 2
    \item Interface 3: Cost = 2
    \item Interface 4: Cost = 3
\end{itemize}

Interfaces 2 and 3 have equal cost, but we must break the tie. In BPDUs, the third number (30 vs 60) is the sender bridge ID. Since 30 < 60, interface 2 provides the best path to the root.

\textbf{Step 3: Determine which interfaces are blocked}

Interface 2 becomes the root port (the port that provides the best path to the root).

For each non-root port, we decide whether to block it based on:
\begin{itemize}
    \item If it receives BPDUs with better paths than what we can offer, we block it
    \item If we can offer a better path, we keep it unblocked (designated port)
\end{itemize}

For interface 1:
\begin{itemize}
    \item It receives BPDU from bridge 20 claiming root 14 with cost 1
    \item Our switch can offer a path to the actual root (12) with cost 3
    \item Since 12 < 14, our path is better, so interface 1 remains unblocked
\end{itemize}

For interfaces 3 and 4:
\begin{itemize}
    \item Both receive BPDUs claiming root 12
    \item For interface 3: received cost is 2, our cost would be 3, so it's better to block
    \item For interface 4: received cost is 3, our cost would be 3, but the sender ID (15) is lower than ours (18), so it's better to block
\end{itemize}

Therefore, interfaces 3 and 4 should be blocked.

\section{Comprehensive Network Analysis Problems}

\subsection{NAT Configuration and Operation}
\textbf{Problem (Exame1\_2018\_correcao.pdf, Problem 2):}

Two private networks are connected to the Internet through NAT routers. Determine NAT table entries and packet header modifications for communication between devices.

\textbf{Solution:}

The scenario involves:
\begin{itemize}
    \item Left private network: PCs A (10.1.1.2) and B (10.1.1.3) behind NAT 1 (public IP 4.1.1.1)
    \item Right private network: PC D (10.1.1.2) and web server F (10.1.1.1) behind NAT 2 (public IP 5.1.1.1)
    \item Public web server E (IP 3.1.1.1)
\end{itemize}

\textbf{Part 1 \& 2: PC A communicating with web server E}

To enable communication between PC A and web server E, we need a NAT entry in NAT 1:

\begin{tabular}{|c|c|c|}
\hline
Internal IP & Internal Port & External Port \\
\hline
10.1.1.2 & 36000 & 40000 \\
\hline
\end{tabular}

When PC A sends a packet to server E, the packet headers change:

\textbf{Packet sent by A:}
\begin{tabular}{|c|c|c|c|}
\hline
Source IP & Destination IP & Source Port & Destination Port \\
\hline
10.1.1.2 & 3.1.1.1 & 36000 & 80 \\
\hline
\end{tabular}

\textbf{Packet received by E:}
\begin{tabular}{|c|c|c|c|}
\hline
Source IP & Destination IP & Source Port & Destination Port \\
\hline
4.1.1.1 & 3.1.1.1 & 40000 & 80 \\
\hline
\end{tabular}

\textbf{Part 3 \& 4: PC B communicating with web server F}

For PC B to communicate with web server F in the other private network, we need NAT entries in both NAT routers:

\textbf{NAT 1 table entry:}
\begin{tabular}{|c|c|c|}
\hline
Internal IP & Internal Port & External Port \\
\hline
10.1.1.3 & 37000 & 41000 \\
\hline
\end{tabular}

\textbf{NAT 2 table entry:}
\begin{tabular}{|c|c|c|}
\hline
Internal IP & Internal Port & External Port \\
\hline
10.1.1.1 & 38000 & 80 \\
\hline
\end{tabular}

When PC B sends a packet to server F, the packet headers change twice:

\textbf{Packet sent by B:}
\begin{tabular}{|c|c|c|c|}
\hline
Source IP & Destination IP & Source Port & Destination Port \\
\hline
10.1.1.3 & 5.1.1.1 & 37000 & 80 \\
\hline
\end{tabular}

\textbf{Packet on the Internet:}
\begin{tabular}{|c|c|c|c|}
\hline
Source IP & Destination IP & Source Port & Destination Port \\
\hline
4.1.1.1 & 5.1.1.1 & 41000 & 80 \\
\hline
\end{tabular}

\textbf{Packet received by F:}
\begin{tabular}{|c|c|c|c|}
\hline
Source IP & Destination IP & Source Port & Destination Port \\
\hline
4.1.1.1 & 10.1.1.1 & 41000 & 38000 \\
\hline
\end{tabular}

This example demonstrates how NAT allows private IPs to communicate through the public Internet, using port translation to multiplex multiple connections through a single public IP address.

\subsection{TCP Congestion Control in a Network with Multiple Flows}
\textbf{Problem (Exame1\_2023Resolucao.pdf, Problem 1):}

Analyze two TCP Reno sessions sharing a link between routers R1 and R2 with capacity 25 Mbit/s and propagation delay 40 ms. Determine bandwidth allocation and behavior.

\textbf{Solution:}

Given:
\begin{itemize}
    \item Link capacity: 25 Mbit/s
    \item Propagation delay: 40 ms (each direction)
    \item R1 buffer size: 250 KB
    \item Two TCP Reno flows sharing the link
\end{itemize}

\textbf{Part 1: Minimum sum of window sizes for full utilization}

For full utilization, the sum of congestion windows must at least equal the bandwidth-delay product (BDP):

\begin{align}
\text{RTT} &= 2 \times \text{propagation delay} = 2 \times 40 \text{ ms} = 80 \text{ ms}\\
\text{BDP} &= \text{Capacity} \times \text{RTT}\\
&= 25 \text{ Mbit/s} \times 0.08 \text{ s}\\
&= 2 \text{ Mb} = 250 \text{ KB}
\end{align}

Therefore, $w_1 + w_2 \geq 250$ KB for full utilization.

\textbf{Part 2: Window size that causes packet loss}

Packet loss occurs when the router buffer fills up. With a 250 KB buffer:

\begin{align}
\text{Queuing delay} &= \frac{\text{Buffer size}}{\text{Capacity}}\\
&= \frac{250 \text{ KB}}{25 \text{ Mbit/s}}\\
&= \frac{250 \times 8 \text{ Kb}}{25 \text{ Mbit/s}}\\
&= 80 \text{ ms}
\end{align}

The total RTT with full buffer becomes:
\begin{align}
\text{RTT with full buffer} &= \text{Base RTT} + \text{Queuing delay}\\
&= 80 \text{ ms} + 80 \text{ ms} = 160 \text{ ms}
\end{align}

The new BDP:
\begin{align}
\text{BDP with full buffer} &= 25 \text{ Mbit/s} \times 0.16 \text{ s}\\
&= 4 \text{ Mb} = 500 \text{ KB}
\end{align}

Therefore, packet loss occurs when $w_1 + w_2 = 500$ KB.

\textbf{Part 3: Effective throughput with packet loss}

When packet loss occurs, TCP's congestion control mechanism activates:
\begin{itemize}
    \item $w_1 + w_2$ oscillates between 250 KB and 500 KB
    \item The link remains fully utilized because even at the minimum, $w_1 + w_2 = 250$ KB equals the BDP
\end{itemize}

Therefore, the effective throughput is 25 Mbit/s (100\% utilization).

\textbf{Part 4: ACK rate}

With segment size 250 bytes:
\begin{align}
\text{Number of segments per second} &= \frac{\text{Throughput}}{\text{Segment size}}\\
&= \frac{25 \times 10^6 \text{ bits/s}}{250 \times 8 \text{ bits/segment}}\\
&= \frac{25 \times 10^6}{2000} \text{ segments/s}\\
&= 12,500 \text{ segments/s}
\end{align}

Each segment generates one ACK, so the ACK rate is 12,500 ACKs per second.

\textbf{Part 5: Long-term bandwidth allocation}

TCP Reno with the same RTT exhibits fairness in bandwidth allocation. The capacity is shared equally among flows:
\begin{itemize}
    \item Flow 1: 12.5 Mbit/s (50\%)
    \item Flow 2: 12.5 Mbit/s (50\%)
\end{itemize}

This is due to the synchronized window adjustments when both flows experience packet loss simultaneously.

\end{document}